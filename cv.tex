\documentclass[letterpaper,12pt]{article}

% set margins
\usepackage[letterpaper, margin=.5in]{geometry}
\raggedright % prevent the indent that starts from the second paragraph in each section

% set font to Times
\usepackage{mathptmx}

% colors
\usepackage[dvipsnames]{xcolor} 
\definecolor{darkblue}{RGB}{61, 90, 128} 

% enables hyperlink and set link color
\usepackage{hyperref} 
\hypersetup{colorlinks=true, urlcolor=darkblue}

% enables itemize
\usepackage[inline]{enumitem}

% tighter spacing than itemize
\setlist[itemize]{align=parleft,left=0pt..1em}
\newenvironment{zitemize}{
\begin{itemize} \vspace{-.8em}\itemsep 0pt \parskip 0pt}
{\end{itemize}\vspace{-.7em}}

% set table left padding to 0
\setlength{\tabcolsep}{0pt}

% section formatting
\usepackage[nostruts]{titlesec}
\titlespacing*{\section}{0em}{0.5em}{0em} % margin left, top, bottom
\titleformat{\section}{\color{darkblue} \scshape \large}{}{0em}{}[\vspace{-0.75em}\hrulefill]

% remove page number
\thispagestyle{empty}

% custom commands
\newcommand{\previousversion}[1]{{\color{gray} [Old Version] #1}}
\newcommand{\todo}[1]{{\color{red} [Todo] #1}}
\newcommand{\narratage}[1]{{\color{blue} [Pang Bai] #1}}
\newcommand{\comment}[1]{{\color{Green} [Comment] #1}}
\newcommand{\proglang}[1]{#1}
% \newcommand{\proglang}[1]{\textbf{#1}}


\begin{document}

% profile
\newcommand{\name}{Yi Guo}
\newcommand{\phone}{(805)895-0554}
\newcommand{\email}{yig053@ucsd.edu}
\newcommand{\address}{43108 Calle Sagrada, Fremont, CA, 94539}
\newcommand{\github}{y1guo}
\newcommand{\linkedin}{y1guo}
\newcommand{\website}{https://y1guo.github.io}

\begin{center}
    \Huge \name \\
    \vspace{1pt}
    \small \phone 
    $|$ \href{mailto:\email}{\underline{\email}} 
    $|$ \address
    % $|$ \href{https://www.linkedin.com/in/\linkedin}{\underline{LinkedIn/\linkedin}} 
    $|$ \href{https://github.com/\github}{\underline{GitHub/\github}} 
    % $|$ \href{\website}{\underline{Home Page}}
    \vspace{-15pt}
\end{center}


\section{Education}

\textbf{University of California, San Diego} \hfill La Jolla, CA \\
\begin{tabular}{p{12em} p{20em}}
    \textit{Ph.D. Physics} 
    & Physics Excellence Award 
\end{tabular}
\hfill Sep 2018 -- Expected Sep 2023

\textbf{University of California, Santa Barbara} \hfill Isla Vista, CA \\
\begin{tabular}{p{12em} p{20em}}
    \textit{B.S. Physics, Mathematics}
    & Academic Honors, Worster Fellowship
\end{tabular}
\hfill Sep 2014 -- Jun 2018


\section{Skills}

\begin{tabular}{p{10em} p{33em}}
    \textbf{Languages and Tools} 
    & Python, C/C++, Fortran, Mathematica, \LaTeX, PyTorch, TensorFlow, Git, Docker \\
    \textbf{Web Development}
    & JavaScript, HTML, CSS, React, React Native, Firebase, Gradio \\
    \textbf{Data Science and ML} 
    & Data Visualization, Parallel/Distributed Computing, Monte Carlo Simulation \\
    & Classification, Regression, Natual Language Processing
\end{tabular}


\section{Software Development Projects}
\textbf{Auto-Transcribe (PyTorch, FFmpeg, Gradio)} \hfill May 2023 -- Aug 2023 \\
\begin{zitemize}
    \item Created a software to automatically transcribe videos, including stem separation and speech recognition. 30X real-time speed each GPU with 85\% accuracy. No comparative products on the market.
    \item Implemented a front-end with \proglang{Gradio} to search, preview and export audios among 10M sentences within 1 sec.
    \item Designed for 24/7 robustness, with error handling and multi-GPU support. Transcribed 50TB within a month.
\end{zitemize}

\textbf{Voice Chatbot (PyTorch, LLM, Gradio)} \hfill Feb 2023 -- Mar 2023 \\
\begin{zitemize}
    \item Developed a chatbot using \proglang{Whisper} for speech recognition, \proglang{GPT-J} as language model, and \proglang{VITS} for text-to-speech. Prototyped a demo with \proglang{Gradio}.
    \item Improved performance of the language model and allow multiple personalities by few-shot learning.
\end{zitemize}

\textbf{Task Tracker App (JavaScript, React, SQL)} \hfill Apr 2022 -- May 2022 \\
\begin{zitemize}
    \item Developed a multi-user Task Tracker App from end to end. Deployed on \proglang{Google Cloud}.
    % \item Designed with \proglang{Material UI}. Adopted \proglang{OAuth2} login system. User data stored in \proglang{Firestore} (\proglang{NoSQL} database).
    \item Used by 20+ family members, friends and colleagues.
\end{zitemize}


\section{Selected Research Projects}
\textbf{Cosmological Parameters Forecast (HPC, Python, Parallel, Distributed)} \hfill Jan. 2022 -- Present \\
\begin{zitemize}
    \item Developed high performance code to predict $f_{\rm NL}$ for various surveys, guiding future LSS survey design.
    \item Optimized the performance to be 10X-100X faster than commonly used packages (e.g. FishLSS).
    \item Increased the sensitivity by $\sim$10X with cutting-edge physics techniques.
\end{zitemize}

\textbf{Standard Model with Axion (HPC, Python, Parallel, Distributed)} \hfill Dec 2020 -- Jan. 2022 \\
\begin{zitemize}
    \item Analyzed the thermal history between axions and Standard Model fermions using quantum field theory. 
    \item Gained SOTA constraints by explicitly solving the Boltzmann equation with 6-dimensional integral.
    \item Speeded up native \proglang{Python} by 100X with \proglang{Numba}. Another 50X from distributed computing with \proglang{Ray}.
\end{zitemize}

\textbf{Spectroscopy and Photometry (ML, Python, C/C++, Fortran, Shell)} \hfill Sep 2015 -- Jun 2017 \\
\begin{zitemize}
    \item Developed a package for rapid spectrum fitting using least squares regression with LMA optimization.
    \item Combined a \proglang{Python} UI for graphics and automation, and a \proglang{Fortran} kernel. Sped up the workflow by $>$10X. 
    \item Created \proglang{C/C++} code and \proglang{Shell} scripts to locate galaxies from images and measure photometry automatically. Reduced time cost by 100X compared to traditional DS9 software.
\end{zitemize}


\section{Publications}

D.~Green, \textbf{Y.~Guo}, J.~Han and B.~Wallisch, (Forthcoming),
``Light Fields during Inflation from Future Galaxy Surveys,''

D.~Green, \textbf{Y.~Guo} and B.~Wallisch,
``Cosmological implications of axion-matter couplings,''
In: \textit{JCAP} 02.02, p. 019 (2022)
DOI: \href{https://iopscience.iop.org/article/10.1088/1475-7516/2022/02/019}{10.1088/1475-7516/2022/02/019}
arXiv: \href{https://arxiv.org/abs/2109.12088?context=hep-ph}{2109.12088 [astro-ph.CO].}


\end{document}

% \vspace{-0.5em}
% \textit{Teaching Assistant} \hfill Oct 2018 -- Present \\
% \begin{zitemize}
%     \item Computational Physics I / II: N-Body Simulation; Quantum Mechanics Simulation. \\
%     \item Reviewed as ``Excellent'' TA by the instructor.
% \end{zitemize}

% \item Calculated the CMB anisotropy phase shift in a neutrino-dominated universe, confirming the series expansion approximation from the photon-dominated scenario is only off by 5\%.