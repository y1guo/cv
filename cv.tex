\documentclass[letterpaper,12pt]{article}

% set margins
\usepackage[letterpaper, margin=.5in]{geometry}
\raggedright % prevent the indent that starts from the second paragraph in each section

% set font to Times
\usepackage{mathptmx}

% colors
\usepackage[dvipsnames]{xcolor} 
\definecolor{darkblue}{RGB}{61, 90, 128} 

% enables hyperlink and set link color
\usepackage{hyperref} 
\hypersetup{colorlinks=true, urlcolor=darkblue}

% enables itemize
\usepackage[inline]{enumitem}

% tighter spacing than itemize
\setlist[itemize]{align=parleft,left=0pt..1em}
\newenvironment{zitemize}{
\begin{itemize} \vspace{-.9em}\itemsep 0pt \parskip 0pt}
{\end{itemize}\vspace{-.5em}}

% set table left padding to 0
\setlength{\tabcolsep}{0pt}

% section formatting
\usepackage[nostruts]{titlesec}
\titlespacing*{\section}{0em}{0.5em}{0em} % margin left, top, bottom
\titleformat{\section}{\color{darkblue} \scshape \large}{}{0em}{}[\vspace{-0.75em}\hrulefill]

% remove page number
\thispagestyle{empty}

% custom commands
\newcommand{\previousversion}[1]{{\color{gray} [Old Version] #1}}
\newcommand{\todo}[1]{{\color{red} [Todo] #1}}
\newcommand{\narratage}[1]{{\color{blue} [Pang Bai] #1}}
\newcommand{\comment}[1]{{\color{Green} [Comment] #1}}
\newcommand{\proglang}[1]{#1}
% \newcommand{\proglang}[1]{\textbf{#1}}


\begin{document}

% profile
\newcommand{\name}{Yi Guo}
\newcommand{\phone}{(805)895-0554}
\newcommand{\email}{yig053@ucsd.edu}
\newcommand{\address}{3813 Camino Lindo, San Diego, CA, 92122}
\newcommand{\github}{y1guo}
\newcommand{\linkedin}{y1guo}
\newcommand{\website}{https://y1guo.github.io}

\begin{center}
    \Huge \name \\
    \vspace{1pt}
    \small \phone 
    $|$ \href{mailto:\email}{\underline{\email}} 
    $|$ \address
    % $|$ \href{https://www.linkedin.com/in/\linkedin}{\underline{LinkedIn/\linkedin}} 
    $|$ \href{https://github.com/\github}{\underline{GitHub/\github}} 
    % $|$ \href{\website}{\underline{Home Page}}
    \vspace{-15pt}
\end{center}


\section{Education}

\textbf{University of California, San Diego} \hfill La Jolla, CA \\
\begin{tabular}{p{12em} p{20em}}
    \textit{Ph.D. Physics} 
    & Physics Excellence Award 
\end{tabular}
\hfill Sep 2018 -- Expected Jun 2023

\textbf{University of California, Santa Barbara} \hfill Isla Vista, CA \\
\begin{tabular}{p{12em} p{20em}}
    \textit{B.S. Physics, Mathematics}
    & Academic Honors, Worster Fellowship
\end{tabular}
\hfill Sep 2014 -- Jun 2018


\section{Skills}

\begin{tabular}{p{10em} p{33em}}
    \textbf{Languages and Tools} 
    & Python, R, Mathematica, C/C++, Fortran, \LaTeX, Pytorch, TensorFlow, Git, Docker \\
    \textbf{Web Development}
    & JavaScript, HTML, CSS, React, React Native, Firebase, Gradio \\
    \textbf{Data Science and ML} 
    & Data Visualization, Parallel/Distributed Computing, Monte Carlo, Regression \\
    & Neural Network, Natual Language Processing
\end{tabular}


\section{Software Development Projects}
\textbf{Machine Learning Application -- AI Voice Assistant} \hfill Feb 2023 -- Mar 2023 \\
\begin{zitemize}
    \item Developed a voice assistant using \proglang{Whisper} for speech recognition, fine-tuned \proglang{GPT-J} as a language model, and \proglang{VITS} for text-to-speech. Prototyped a demo with \proglang{Gradio}.
\end{zitemize}

\textbf{Web Application -- Personal Task Management System} \hfill Apr 2022 -- May 2022 \\
\begin{zitemize}
    \item Built a Progressive Web App on \proglang{Firebase} for personal task tracking using \proglang{React} and \proglang{Material UI}. Implemented \proglang{NoSQL} database and \proglang{OAuth2} login.
\end{zitemize}


\section{Experience}
\textbf{University of California San Diego} \hfill La Jolla, CA \\
\textit{Research Assistent} \hfill Dec 2019 -- Present \\
\begin{zitemize}
    \item Developed cutting-edge numerical techniques to predict the sensitivities of primordial non-gaussianity for upcoming LSS surveys, incorporating the kSZ effect and utilizing a multi-tracer approach. This innovative method resulted in a remarkable reduction of uncertainty by an order of magnitude for future surveys.
    \item Quantitatively analyzed the constraints on the coupling strength between axions and standard model fermions using tree level quantum field theory and modern cosmology. Adopted \proglang{Numba} to speed up parallel \proglang{Python} by 100X. Work published on \textit{JCAP}.
    \item Calculated the CMB anisotropy phase shift in a neutrino-dominated universe, confirming the series expansion approximation from the photon-dominated scenario is only off by 5\%.
\end{zitemize}
\vspace{-0.5em}
\textit{Teaching Assistant} \hfill Oct 2018 -- Present \\
\begin{zitemize}
    \item Computational Physics I / II: N-Body Simulation; Quantum Mechanics Simulation. \\
    \item Reviewed as ``Excellent'' TA by the instructor.
\end{zitemize}

\textbf{University of California Santa Barbara} \hfill Isla Vista, CA \\
\textit{Undergraduate Researcher} \hfill Sep 2015 -- Jun 2017 \\
\begin{zitemize}
    \item Presented research on gas behavior in galaxy mergers at UCSB undergraduate symposium and Worster symposium with an audience of 200 faculty and students.
    \item Developed a software package in \proglang{Python} and \proglang{Fortran} that improved the spectrum fitting workflow by 10X. Created \proglang{C} codes and \proglang{Shell} scripts to auto-locate galaxies and measure photometries.
\end{zitemize}


\section{Publications}

D.~Green, \textbf{Y.~Guo}, J.~Han and B.~Wallisch, (Forthcoming),
``Light Fields during Inflation from Future Galaxy Surveys,''

D.~Green, \textbf{Y.~Guo} and B.~Wallisch,
``Cosmological implications of axion-matter couplings,''
In: \textit{JCAP} 02.02, p. 019 (2022)
DOI: \href{https://iopscience.iop.org/article/10.1088/1475-7516/2022/02/019}{10.1088/1475-7516/2022/02/019}
arXiv: \href{https://arxiv.org/abs/2109.12088?context=hep-ph}{2109.12088 [astro-ph.CO].}


\end{document}