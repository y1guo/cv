\documentclass[letterpaper,12pt]{article}

% set margins
\usepackage[letterpaper, margin=.5in]{geometry}
\raggedright % prevent the indent that starts from the second paragraph in each section

% set font to Times
\usepackage{mathptmx}

% colors
\usepackage[dvipsnames]{xcolor} 
\definecolor{darkblue}{RGB}{61, 90, 128} 

% enables hyperlink and set link color
\usepackage{hyperref} 
\hypersetup{colorlinks=true, urlcolor=darkblue}

% enables itemize
\usepackage[inline]{enumitem}

% tighter spacing than itemize
\setlist[itemize]{align=parleft,left=0pt..1em}
\newenvironment{zitemize}{
\begin{itemize} \vspace{-.9em}\itemsep 0pt \parskip 0pt}
{\end{itemize}\vspace{-.5em}}

% set table left padding to 0
\setlength{\tabcolsep}{0pt}

% section formatting
\usepackage[nostruts]{titlesec}
\titlespacing*{\section}{0em}{0.5em}{0em} % margin left, top, bottom
\titleformat{\section}{\color{darkblue} \scshape \large}{}{0em}{}[\vspace{-0.75em}\hrulefill]

% remove page number
\thispagestyle{empty}

% custom commands
\newcommand{\previousversion}[1]{{\color{gray} [Old Version] #1}}
\newcommand{\todo}[1]{{\color{red} [Todo] #1}}
\newcommand{\narratage}[1]{{\color{blue} [Pang Bai] #1}}
\newcommand{\comment}[1]{{\color{Green} [Comment] #1}}
\newcommand{\proglang}[1]{#1}
% \newcommand{\proglang}[1]{\textbf{#1}}


\begin{document}

% profile
\newcommand{\name}{Yi Guo}
\newcommand{\phone}{(805)895-0554}
\newcommand{\email}{yig053@ucsd.edu}
\newcommand{\address}{3813 Camino Lindo, San Diego, CA, 92122}
\newcommand{\github}{y1guo}
\newcommand{\linkedin}{y1guo}
\newcommand{\website}{https://y1guo.github.io}

\begin{center}
    \Huge \name \\
    \vspace{1pt}
    \small \phone 
    $|$ \href{mailto:\email}{\underline{\email}} 
    $|$ \address
    % $|$ \href{https://www.linkedin.com/in/\linkedin}{\underline{LinkedIn/\linkedin}} 
    $|$ \href{https://github.com/\github}{\underline{GitHub/\github}} 
    % $|$ \href{\website}{\underline{Home Page}}
    \vspace{-15pt}
\end{center}


\section{Education}

\textbf{University of California, San Diego} \hfill La Jolla, CA \\
\begin{tabular}{p{12em} p{20em}}
    \textit{Ph.D. Physics} 
    & Physics Excellence Award 
\end{tabular}
\hfill Sep 2018 -- Expected Jun 2023

\textbf{University of California, Santa Barbara} \hfill Isla Vista, CA \\
\begin{tabular}{p{12em} p{20em}}
    \textit{B.S. Physics, Mathematics}
    & Academic Honors, Worster Fellowship
\end{tabular}
\hfill Sep 2014 -- Jun 2018


\section{Skills}

\begin{tabular}{p{10em} p{33em}}
    \textbf{Languages and Tools} 
    & Python, Mathematica, C/C++, Fortran, \LaTeX, Jupyter, Pytorch, Git, Docker \\
    \textbf{Web Development}
    & JS, HTML, CSS, React, React Native, Firebase, Gradio \\
    \textbf{Numerical Research} 
    & Data Visualization, Parallel/Distributed Computing, Monte Carlo, Regression \\
\end{tabular}



\section{Experience}
\textbf{University of California San Diego} \hfill La Jolla, CA \\
Teaching Assistant \hfill Oct 2018 -- Present \\
\begin{zitemize}
    \item Computational Physics I/II: N-Body Simulation; Quantum Mechanics Simulation. \\
    \item Reviewed as ``Excellent'' by the instructor.
\end{zitemize}
\vspace{-0.5em}
Research Assistent \hfill Dec 2019 -- Present \\
\begin{zitemize}
    \item Numerically forecasted the sensitivities of the primordial non-gaussianity for future LSS surveys. Innovated to include the kSZ effect, and adopt multi-tracer technique. \todo{As a result, greatly reduced the uncertainty by ???.}
    \item Quantitatively analyzed the constraints on the coupling strength between axions and standard model fermions, with tree level quantum field theory and modern cosmology. Adopted \proglang{Numba} to speed up parallel \proglang{Python} by 100X. Work published on \textit{JCAP}.
    \item Calculated the CMB anisotropy phase shift in neutrino dominated universe. Confirmed the series expansion from photon dominated universe is only off by 5\%.
\end{zitemize}

\textbf{University of California Santa Barbara} \hfill Isla Vista, CA \\
Student Researcher \hfill Sep 2015 -- Jun 2017 \\
\begin{zitemize}
    \item Presented research of gas behavior in galaxy mergers on UCSB undergraduate symposium and Worster Symposium with 200 from UCSB in the audience.
    \item Created package in \proglang{Python} and \proglang{Fortran} that speeds up spectrum fitting workflow by 10X. Created \proglang{C} codes and \proglang{SHELL} scripts to audo-locate galaxies and measure photometries.
\end{zitemize}



\section{Software Development Projects}

\textbf{Web Application -- Personal Task Management System} \hfill Apr 2022 -- May 2022 \\
\begin{zitemize}
    \item Developing a voice assistant, with \proglang{Whisper} as speech recognition, fine-tuned \proglang{GPT-J} as language model and \proglang{VITS} as text-to-speech. Demo prototyped with \proglang{Gradio}.
    \item Developed a Progressive Web App on \proglang{Firebase} for personal task tracking. Built with \proglang{React} and \proglang{Material UI}, using \proglang{no-SQL} database and \proglang{OAuth2} login. 
\end{zitemize}


\section{Publications}

D.~Green, \textbf{Y.~Guo}, J.~Han and B.~Wallisch, (Forthcoming),
``Light Fields during Inflation from Future Galaxy Surveys,''

D.~Green, \textbf{Y.~Guo} and B.~Wallisch,
``Cosmological implications of axion-matter couplings,''
In: \textit{JCAP} 02.02, p. 019 (2022)
DOI: \href{https://iopscience.iop.org/article/10.1088/1475-7516/2022/02/019}{10.1088/1475-7516/2022/02/019}
arXiv: \href{https://arxiv.org/abs/2109.12088?context=hep-ph}{2109.12088 [astro-ph.CO].}


\end{document}