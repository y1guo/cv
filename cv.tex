\documentclass[letterpaper,12pt]{article}

% set margins
\usepackage[letterpaper, margin=.5in]{geometry}
\raggedright % prevent the indent that starts from the second paragraph in each section

% set font to Times
\usepackage{mathptmx}

% colors
\usepackage[dvipsnames]{xcolor} 
\definecolor{darkblue}{RGB}{61, 90, 128} 

% enables hyperlink and set link color
\usepackage{hyperref} 
\hypersetup{colorlinks=true, urlcolor=black}

% external link icon
\usepackage{tikz}
\newcommand{\ExternalLink}{
    \tikz[x=1.5ex, y=1.5ex, baseline=0ex, color=darkblue]{
        \begin{scope}[x=1.3ex, y=1.3ex]
            \clip (-0.1,-0.1) 
                --++ (-0, 1.2) 
                --++ (0.6, 0) 
                --++ (0, -0.6) 
                --++ (0.6, 0) 
                --++ (0, -1);
            \path[draw, 
                line width = 1, 
                rounded corners=0.5] 
                (0,0) rectangle (1,1);
        \end{scope}
        \path[draw, line width = 1] (0.5, 0.5) 
            -- (1, 1);
        \path[draw, line width = 1] (0.6, 1) 
            -- (1, 1) -- (1, 0.6);
    }
}

% enables itemize
\usepackage[inline]{enumitem}

% tighter spacing than itemize
\setlist[itemize]{align=parleft,left=0pt..1em}
\newenvironment{zitemize}{
\begin{itemize} \vspace{-.8em}\itemsep 0pt \parskip 0pt}
{\end{itemize}\vspace{-.7em}}

% set table left padding to 0
\setlength{\tabcolsep}{0pt}

% section formatting
\usepackage[nostruts]{titlesec}
\titlespacing*{\section}{0em}{0em}{0em} % margin left, top, bottom
\titleformat{\section}{\vspace{-0.2em}\color{darkblue} \scshape \large}{}{0em}{}[\vspace{-1.0em}\hrulefill\vspace{-0.2em}]

% remove page number
\thispagestyle{empty}

% custom commands
\newcommand{\previousversion}[1]{{\color{gray} [Old Version] #1}}
\newcommand{\todo}[1]{{\color{red} [Todo] #1}}
\newcommand{\narratage}[1]{{\color{blue} [Pang Bai] #1}}
\newcommand{\comment}[1]{{\color{Green} [Comment] #1}}
\newcommand{\proglang}[1]{#1}
% \newcommand{\proglang}[1]{\textbf{#1}}


\begin{document}

% profile
\newcommand{\name}{Yi Guo}
\newcommand{\phone}{(805)895-0554}
% \newcommand{\email}{yig053@ucsd.edu}
\newcommand{\email}{guoyi0328@gmail.com}
\newcommand{\address}{43108 Calle Sagrada, Fremont, CA, 94539}
\newcommand{\github}{y1guo}
\newcommand{\linkedin}{y1guo}
\newcommand{\website}{https://y1guo.github.io}

\begin{center}
    \Huge \name \\
    \vspace{1pt}
    \small \phone 
    $|$\email
    $|$\address
    % $|$ \href{https://www.linkedin.com/in/\linkedin}{\underline{LinkedIn/\linkedin}} 
    $|$\href{https://github.com/\github}{GitHub/\github}
    % $|$ \href{\website}{\underline{Home Page}}
    \vspace{-10pt}
\end{center}


% \section{Summary}

% A passionate PhD candidate in Physics with strong quantitative and analytical skills in problem solving, \\
% 9+ years of programming techniques, and hands-on experience in data science and machine learning algorithms.


\vspace{-0.9em}
\section{Skills}
\vspace{0.3em}

\begin{tabular}{p{7em} p{38em}}
    \textbf{Languages}
    & Python, JavaScript (Next.js, React), Rust, Java, C/C++, Swift \\
    \textbf{DB \& DevOps}
    & PostgreSQL, Redis, SQLite, Qdrant, GCP, Firebase, Docker, K8s, Git, GitHub \\
    \textbf{AI \& ML}
    & OpenAI, Anthropic, Gemini, Deepgram, Cartesia, WhisperX, XTTS, Silero-VAD, Llama3.1 \\
    & Embeddings, RAG, Prompting, Few-shot Learning, Tool-use, NLP \\
    \textbf{Technologies}
    & RESTful APIs, WebSockets, Distributed Computing, Data Visualization, GitHub Copilot \\
    % \textbf{Machine Learning}
    % & Transformers, CNNs, GANs, VAEs, Diffusion Models, NLP, ASR \\
    % \textbf{Data Science}
    % & Data Visualization, Monte Carlo Simulation, Parallel/Distributed Computing \\
\end{tabular}


\vspace{0.1em}
\section{Professional Experience}

\textbf{RealChar, Inc. - Software Engineer} \hfill Aug 2023 -- Present \\
\begin{zitemize}
    \item[~]\leavevmode \textbf{\href{https://heyrevia.ai/}{Project: Revia \ExternalLink}}
    \item Reduced voice chat e2e latency to an industry-leading 500ms by adopting the architecture of self-driving cars along with a multi-agent system and frequent caching.
    \item Prototyped the MVP with Python, React, PostgreSQL, Firebase, Twilio and Agora.
    \item Major contributor of the perception, prediction, control and planner modules.
    \item Developed a websocket server for Agora Java SDK to communicate with the python backend.
    \item Designed and implemented RESTful APIs for core functionalities including calls, call histories and CRUD.

    \item[~]\leavevmode \textbf{\href{https://rebyte.ai/}{Project: Rebyte \ExternalLink}}
    \item Identified and resolved backend latency bottlenecks by addressing issues within the Deno sandbox.
    \item Expanded support for additional LLM and embedding providers, integrating new tokenizers.

    \item[~] \leavevmode\textbf{\href{https://github.com/Shaunwei/RealChar}{Project: RealChar (Open Source) \ExternalLink}}
    \item Major contributor of key features such as RAG, phone call mode, meeting mode, and chat on image.
    \item Implemented RESTful API servers for WhisperX and XTTS, containerized, and deployed on GCE.
    % \item Reduced STT TTFT by 8x (vs faster-whisper) and TTS TTFB by 4x (vs ElevenLabs).
    \item Achieved an 8x reduction in speech-to-text latency (compared to faster-whisper) and a 4x reduction in text-to-speech latency (compared to ElevenLabs).
    \item Built a server-side VAD using Silero-VAD and loudness algorithms, foundational for the phone call mode.
\end{zitemize}


\textbf{Personal Projects}
\begin{zitemize}
    \item[~] \textbf{\href{https://github.com/y1guo/auto-transcribe}{Auto-Transcribe - Open Source Developer \ExternalLink}} \hfill May 2023 -- Aug 2023
    \item Created software for automatic video transcription, incorporating stem separation and speech recognition.
    \item Delivered 30x real-time speed, enabling 24/7 unattended transcription with alignment.
    \item Developed a web UI capable of searching through 10 million audio segments by transcript within 1s.
    \item Ensured continuous operation with crash recovery, multi-GPU support. Transcribed 50TB of video in 1 month.
    \item Leveraged the tool to produce video content, garnering over 680,000 views and 1,400 engaged subscribers.
\end{zitemize}


% \textbf{\href{https://github.com/y1guo/chatbot}{Voice Chatbot (PyTorch, LLM) \ExternalLink}} \hfill Feb 2023 -- Mar 2023 \\
% \begin{zitemize}
%     \item Developed a chatbot using \proglang{Whisper} for speech recognition, \proglang{GPT-J} as language model, and \proglang{VITS} for text-to-speech. Prototyped a demo with \proglang{Gradio}.
%     \item Improved the performance of the language model and allowed multiple personalities by few-shot learning.
% \end{zitemize}

% \textbf{\href{https://github.com/y1guo/system}{Task Tracker App (HTML, JavaScript, React) \ExternalLink}} \hfill Apr 2022 -- May 2022 \\
% \begin{zitemize}
%     \item Developed a multi-user Task Tracker App from end to end. Deployed on \proglang{Google Cloud Platform}.
%     % \item Designed with \proglang{Material UI}. Adopted \proglang{OAuth2} login system. User data stored in \proglang{Firestore} (\proglang{NoSQL} database).
%     \item Used by 10+ family members, friends and colleagues.
% \end{zitemize}


\section{Research Experience}

\textbf{High-Performance Computing for Cosmology} \hfill Jan 2022 -- Nov 2023 \\
\begin{zitemize}
    \item Developed high-performance code calculating cosmological parameters, achieving speeds $\sim$100x faster than existing packages like FishLSS. Improved forecasted sensitivity by $\sim$10x using advanced physics techniques.
\end{zitemize}

\textbf{Axion Interaction Constraints} \hfill Dec 2020 -- Jan 2022 \\
\begin{zitemize}
    \item Accelerated computations by compiling Python to C (100x speedup) and by implementing distributed computing with Ray. Computed the thermal history to place leading constraints on axion-fermion interactions.
\end{zitemize}

% \textbf{Spectroscopy and Photometry (ML, Python, C/C++, Fortran, Shell)} \hfill Sep 2015 -- Jun 2017 \\
% \begin{zitemize}
%     \item Developed a spectrum fitting package (10x workflow speed up) and a galaxy photometry pipeline (100x).
% \end{zitemize}


\section{Education}

\textbf{University of California, San Diego} \hfill La Jolla, CA \\
\begin{tabular}{p{12em} p{20em}}
    \textit{Ph.D. Physics} 
    & Physics Excellence Award 
\end{tabular}
\hfill Sep 2018 -- Dec 2023

\textbf{University of California, Santa Barbara} \hfill Isla Vista, CA \\
\begin{tabular}{p{12em} p{20em}}
    \textit{B.S. Physics, Mathematics}
    & Academic Honors, Worster Fellowship
\end{tabular}
\hfill Sep 2014 -- Jun 2018


\section{Selected Publications}

D.~Green, \textbf{Y.~Guo}, J.~Han and B.~Wallisch,
``Light Fields during Inflation from BOSS and Future Galaxy Surveys,''
In: \textit{JCAP} 05, p. 090 (2024)
DOI: \href{https://iopscience.iop.org/article/10.1088/1475-7516/2024/05/090}{10.1088/1475-7516/2024/05/090}
arXiv: \href{https://arxiv.org/abs/2311.04882}{2311.04882 [astro-ph.CO].}

D.~Green, \textbf{Y.~Guo} and B.~Wallisch,
``Cosmological Implications of Axion-Matter Couplings,''
In: \textit{JCAP} 02.02, p. 019 (2022)
DOI: \href{https://iopscience.iop.org/article/10.1088/1475-7516/2022/02/019}{10.1088/1475-7516/2022/02/019}
arXiv: \href{https://arxiv.org/abs/2109.12088?context=hep-ph}{2109.12088 [astro-ph.CO].}


\end{document}

% \vspace{-0.5em}
% \textit{Teaching Assistant} \hfill Oct 2018 -- Present \\
% \begin{zitemize}
%     \item Computational Physics I / II: N-Body Simulation; Quantum Mechanics Simulation. \\
%     \item Reviewed as ``Excellent'' TA by the instructor.
% \end{zitemize}

% \item Calculated the CMB anisotropy phase shift in a neutrino-dominated universe, confirming the series expansion approximation from the photon-dominated scenario is only off by 5\%.